\documentclass{article}
\usepackage[MeX]{polski}
\usepackage[utf8]{inputenc}
\usepackage{amsfonts}
\usepackage{amssymb}
\usepackage{graphicx}
\usepackage[pdftex]{color,graphicx}

\author{Piotr Dobrowolski}
\title{Raport \\{\small Projekt Zaliczeniowy z Rachunku Prawdopodobieństwa i Statystyki} }
\date{\today}
\frenchspacing

\begin{document}
\maketitle
\tableofcontents
\section{Model}
\section{O testowaniu}
\subsection{Założenia}
Testowanie strategii w moim projekcie odbywa się poprzez wielokrotne symulowanie kupowania
obiadów przez klientów. Podejście do testowania: 1-wizualizajca danej, konkretnej, symulacji, oraz 2-
wielokrotne symulowanie - szukanie najgorszych danych wejściowych.
\subsection{Symulacja}
\paragraph{symulacja = function(M, k, Grupy, strategia, wypisuj = 0)::} Dla danych parametrów M, k, Grupy, strategia
funkcja przeprowadza symulacje kupowania obiadów przez klientów. Każdy klient (wszystkich jest N) przychodzi codziennie
do bufetu i kupuje, bądź też nie, obiad. Każdego dnia pyta się strategii o koszt, a pod koniec dnia przekazuje
strategii ilość kupionych obiadów.
\subsection{Konkretna symulacja - testowanie}
Testowanie strategii na jednej, konkretnej symulacji. Polega na porównaniu działania strategii do strategii optymalnej
i losowaj, gdzie optymalna (opisana później) stanowi pewne maksimum do którego chcę, aby dążyła (w przeciągu dni) testowana strategia
, a losowa minimum. 
\paragraph{test\_LO = function(M, k, Grupy, strategia, metoda\_symulacji)::} Funkcja tworzy wykresy porównujące wybraną
strategię do optymalnej i losowej. Dzięki temu można przeanalizować jak w czasie (dni) działa dana strategia. Wypisywane jest także
podsumowanie zysków każdego dnia, oraz zysk całkowity, uzyskany przez M dni.
\subsection{Grupy testów}
\paragraph{testy\_szukaj\_najgorszych\_M = function(Grupy, k, strategia, metoda\_symulacji, wypisuj = 0)::} Funkcja testuje strategię
ze względu na radzenie sobie z różną ilością dni. Chodzi o to żeby strategia jak najszybciej umożliwiała dobrać
najlepszy koszt.
\paragraph{testy\_szukaj\_najgorszych = function(strategia, metoda\_symulacji, wypisuj = 0)::} Funkcja wykożystuje tę powyższą, ale teraz
wykonuje testy dla różnych grup klientów, które dobiera losowo. Dzięki niej można określić (i zmieniając parametry funkcji powyższej
, czyli ilości dni jakie będą testowane) dla jakich ilości dostępnych dni, oraz dla jakich grup strategia najgorzej sobie radzi.
\\ Dzięki tej funkcji (przy parametrze wypisuj > 0) można także uzystać informacje o atomowych testach występujących w tej grupie testów
. Dzięki temu można przeanalizować daną strategię.

\section{Strategie}
Poniżej znajduje się 
\subsection{Strategie do testowania}
\paragraph{Strategia - losuję koszt każdego dnia}
\paragraph{Strategie - optymalna, znam grupy}
\subsection{Strategia szukam}

\section{Testowanie}
\subsection{Wynik 1}

\section{Wnioski}
\end{document}
